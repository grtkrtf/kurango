\documentclass[12pt]{article}
	%FORMATTING
		\usepackage{graphicx} %op shit
		\usepackage[margin=1in]{geometry} %Set margins
		\usepackage{fixltx2e} %subscript
	%LINGUISTICS-RELATED PACKAGES
		\usepackage{qtree} %syntax trees
		\usepackage{covington} %interlinearizations
		\usepackage{amsmath,amsthm,amssymb} %empty set and some other shit
		\usepackage{tipa} %ipa

	\let\oldemptyset\emptyset %Redefines the shitty empty set symbol to the old one.
	\let\emptyset\varnothing

	%stuff for handling how awful it is to use default tipa syntax
		\newcommand{\R}{\textipa{R}} %alv tap
		\newcommand{\N}{\textipa{N}} %eng
		\newcommand{\OO}{\textipa{O}} %open o, DONT FORGET THE SECOND O!!!
		\newcommand{\B}{\textipa{B}} %bilab fric voi
		\newcommand{\G}{\textipa{G}} %velar fric voi
		\newcommand{\W}{\textturnmrleg} %velar approximant
		%\textipa{:}

		\newcommand{\0}{$\emptyset$} %Null
		\newcommand{\D}{\scshape} %smallcaps

		\newcommand{\kurango}{K\textipa{uRaNO}}
		\newcommand{\nom}{\D{nom}}
		\newcommand{\erg}{\D{erg}}
		\newcommand{\abs}{\D{abs}}
		\newcommand{\gen}{\D{gen}}
		\newcommand{\poss}{\D{poss}}
		\newcommand{\emo}{\D{emo}}
		\newcommand{\cop}{\D{cop}}
		\newcommand{\caus}{\D{caus}}
		\newcommand{\pst}{\D{pst}}
		\newcommand{\fut}{\D{fut}}

%--------------------------------------------------------------%
%  EXAMPLE INTERLINEAR GLOSS, ADAPTED FROM LEV'S DISSERTATION  
%															   
%	\begin{example}											   
%	\label{ex:nasalpoa}										   
%		Ontagake. \textipa{[ontagakse]}						   
%		\gll o= n- tag -ak -e 								   
%		3nmS= irreal burn -perf -irreal.i 					   
%		\glt `It will burn.' 								   
%		\glend 												   
%	\end{example} 											   
%  															   
%	%Reference examples like this: (\ref{ex:nasalpoa})         
%--------------------------------------------------------------%

%This is a test\footnote{This is not a test; it's an example of a footnote.}

\begin{document}
\title{questions for lev}
\maketitle
\begin{enumerate}
	
	\item Case assignment\\\kurango 's case assignment system is what I think is called ``Fluid-S''---S patterns with A and O depending on volitionality of the agent.
		\begin{example}
		\label{ex:case_erg}
			Nakari. [naka\R i]
			\gll na- \0- ka\R i
			{\D{1}}- {\D{erg}}- sleep
			\glt `I sleep.' (by my own volition.)
			\glend
		\end{example}

		\begin{example}
		\label{ex:case_abs}
			Nakukari. [nakuka\R i]
			\gll na- ku- ka\R i
			{\D{1}}- {\D{nom}}- sleep
			\glt `I sleep.' (no entailment about my volition.)
			\glend
		\end{example}

		\begin{example}
			\label{ex:nonapplicative}
				Nimotivagaana. \textipa{[nimOtiBaga:na]}
				\gll ni- {\0-} m\OO ti -\B a -ga\textipa{:} -na -\0
				2- {\erg-} die -{\D{caus}} -{\D{fut}} -1 -\abs
				\glt `You are going to kill me.' (More on causatives later...)
				\glend
			\end{example}

	In transitive sentences, ERG and ABS case are used. NOM and ABS case are marked the same, but one marks S\textsubscript{o} and one marks O. Should I even call it NOM in (\ref{ex:case_erg}) or should I just call it ABS?

	If you negate a Fluid-S intransitive sentence, are you negating the volition (or lack thereof) or are you negating the action of the verb? How would you negate the other one?

	\item Case assignment in ditransitives\\In a ditransitive sentence, what is the case of the verb's second argument? Is it GEN? OBL? DAT? What are GEN, OBL, and DAT? Are there other kinds of case?
		\begin{example}
			\label{ex:case_ditrans}
				Nawivinu namiromiku nungu. \textipa{[na\W i\B inu nami\R\OO miku nu\N u]}
				\gll na- \0- \W i\B i -nu {-\0} na- {mi\R\OO} -mi -ku nu -\N u
				1- \erg- give -3 {-\abs} 1- cat {-\poss} {-\abs} 3 -\gen
				%is case assignment in this sentence OK?
				\glt `I give him/her my cat.'
				\glend
		\end{example}

	\item Causatives\\Below is an example of how causatives work in \kurango . Are causatives usually marked on the verb? If they are, is my form typologically acceptable?

			\begin{example}
			\label{ex:noncausative}
				Nakumotigaa. \textipa{[nakumOtiga:]}
				\gll na- ku- m\OO ti -ga\textipa{:}
				{\D{1}}- {\D{nom}}- die -{\D{fut}}
				\glt `I am going to die.'
				\glend
			\end{example}

			\begin{example}
			\label{ex:causative}
				Namotivagaana. \textipa{[namOtiBaga:na]}
				\gll na- {\0-} m\OO ti -\B a -ga\textipa{:} -na -\0
				{\D{1}} {\D{erg}}- die -{\D{caus}} -{\D{fut}} -1 -\abs
				\glt `I am going to kill myself.' (lit: `I am going to make myself die.')
				\glend
			\end{example}

	Also, how do causatives work in a Fluid-S system? Is the case assignment on O (previously ``underlyingly?'' S) determined by volition, or does it follow standard rules for transitive sentence case assignment?

			\begin{example}											   
				\label{ex:fluids_erg}										   
						Gangaka nasivamonu. \textipa{[gaNaka nasiBamOnu]}
						\gll ga -\N a -ka na- \0- si -\B a {-m\OO} -nu -\0						   
			   			{\emo} -{\D{neg}} -{\D{emo:happy}} 1- {\erg-} {\cop} {-\caus} {-\pst} 3 -\abs
						\glt `I made him/her unhappy.' (It is my fault that he/she was unhappy.) 								   
						\glend 												   
				\end{example}

				\begin{example}
				\label{ex:fluids_abs}
						Gangaka nakusivamonu. \textipa{[gaNaka nakusiBamOnu]}
						\gll ga -\N a -ka na- ku- si -\B a {-m\OO} -nu -\0				   
			   			{\emo} -{\D{neg}} -{\D{emo:happy}} 1- {\nom}- {\cop} {-\caus} {-\pst} 3 -\abs
						\glt `I made him/her unhappy.' (No implication that it was my fault.)
						\glend
				\end{example}

	What about the causative morpheme without the introduction of another argument? Do languages do that? Is below typologically acceptable?

		\begin{example}
		 	\label{ex:noncausative}
		 		Nakumotivagaa. \textipa{[nakumOtiBaga:]}
		 		\gll na- ku- m\OO ti -\B a -ga\textipa{:}
		 		1- {\nom-} die {-\caus} {-\fut}
		 		\glt `I am going to be killed' (lit: I am going to be made dead) %Is this future perfect or just future? wtf
		 		\glend
		 \end{example}

	How do you negate the causative versus negating the verb?

	\item Applicatives\\How are applicatives usually marked? A suffix on the introduced argument or a suffix on the verb?

			\begin{example}
			\label{ex:nonapplicative}
				Nimotivagaana. \textipa{[nimOtiBaga:na]}
				\gll ni- {\0-} m\OO ti -\B a -ga\textipa{:} -na -\0
				2- {\erg-} die -{\D{caus}} -{\D{fut}} -1 -\abs
				\glt `You are going to kill me.'
				\glend
			\end{example}

			\begin{example}
			\label{ex:applicative}
				Nimotivagaana soodongu'uti. \textipa{[nigu naku mOtiBaga: sO:dONuPuti]}
				\gll ni- {\0-} m\OO ti -\B a -ga\textipa{:} -na -\0 {s\OO\textipa{:}d\OO} -\N u -\textipa{P}uti
				1- {\erg}- kill -{\caus} -{\fut} -1 -{\abs} sword {-\gen} -{\D{appl}}
				\glt `You are going to kill me with a sword.' (lit: `You made me die with a sword.')
				\glend
			\end{example}

	negation in applicatives\\applicatives without another argument

	\item Antipassives\\What are they and how could I implement them?

	\item Directionals\\Let's discuss directionals and how I could use them instead of prepositions (like in the sentence i am going to the store, the directional would be marked on `store')

	im going to jump versus im going-dir:prox jump 'im going to go jump'
	im going to jump versus im going to jump store-dir:prox `im going to jump to the store' (Maybe use an applicative here on store as well as it is an introduced argument?)

	\item How do you draw origo graphs?


\end{enumerate}
\end{document}