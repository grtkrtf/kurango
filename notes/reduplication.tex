\documentclass[12pt]{article}
	%FORMATTING
		\usepackage{graphicx} %op shit
		\usepackage[margin=1in]{geometry} %Set margins
		\usepackage{fixltx2e} %subscript
	%LINGUISTICS-RELATED PACKAGES
		\usepackage{qtree} %syntax trees
		\usepackage{covington} %interlinearizations
		\usepackage{amsmath,amsthm,amssymb} %empty set and some other shit
		\usepackage{tipa} %ipa
		\usepackage{float} %resizing tables

	\let\oldemptyset\emptyset %Redefines the shitty empty set symbol to the old one.
	\let\emptyset\varnothing

	%stuff for handling how awful it is to use default tipa syntax
		\newcommand{\R}{\textipa{R}} %alv tap
		\newcommand{\N}{\textipa{N}} %eng
		\newcommand{\OO}{\textipa{O}} %open o, DONT FORGET THE SECOND O!!!
		\newcommand{\B}{\textipa{B}} %bilab fric voi
		\newcommand{\G}{\textipa{G}} %velar fric voi
		\newcommand{\W}{\textipa{V}} %velar approximant
		\newcommand{\F}{\textipa{F}} %-voi bilab fric
		\newcommand{\glot}{\textipa{P}}
		%\textipa{:}

		\newcommand{\0}{$\emptyset$} %Null
		\newcommand{\D}{\scshape} %smallcaps

		\newcommand{\kurango}{K\textipa{uRaNO}}
		\newcommand{\nom}{\D{nom}}
		\newcommand{\erg}{\D{erg}}
		\newcommand{\abs}{\D{abs}}
		\newcommand{\gen}{\D{gen}}
		\newcommand{\poss}{\D{poss}}
		\newcommand{\emo}{\D{emo}}
		\newcommand{\cop}{\D{cop}}
		\newcommand{\caus}{\D{caus}}
		\newcommand{\pst}{\D{pst}}
		\newcommand{\fut}{\D{fut}}
		\newcommand{\negative}{\D{neg}}
		\newcommand{\interr}{\D{interr}}
		\newcommand{\reas}{\D{reas}}
		\newcommand{\dirprox}{\D{dir:prox}}

%--------------------------------------------------------------%
%  EXAMPLE INTERLINEAR GLOSS, ADAPTED FROM LEV'S DISSERTATION  
%															   
%	\begin{example}											   
%	\label{ex:nasalpoa}										   
%		Ontagake. \textipa{[ontagakse]}						   
%		\gll o= n- tag -ak -e 								   
%		3nmS= irreal burn -perf -irreal.i 					   
%		\glt `It will burn.' 								   
%		\glend 												   
%	\end{example} 											   
%  															   
%	%Reference examples like this: (\ref{ex:nasalpoa})         
%--------------------------------------------------------------%

%This is a test\footnote{This is not a test; it's an example of a footnote.}

\begin{document}
\title{Reduplication in \kurango}
\maketitle

	One of \kurango 's hallmarks is its reduplicative system. It has multiple reduplicative processes.

	\section{Intensification}
		 Semantically, this reduplicative process results in an intensification of the meaning of the morpheme. This more or less restricts where reduplication can or can not occur---it is common on intensifiable parts of speech like adjectives, and very rare on less-intensifiable parts of speech like verbs. The reduplicative template varies based on how many syllables\footnote{Syllable being one vowel and its onset, or one grapheme in the \emph{kuraito} orthography.} are in the stem.

		\subsection{One and two-syllable stems}
			One and two-syllable stems are copied and reduplicated entirely\footnote{Forms in these tables ignore phonological rules and represent underlying forms.}.
				\begin{table}[H]
				\centering
					\begin{tabular}{ll}
					Original Form & Reduplicated Form \\ \hline\hline
					ga -ti -xu si & ga -titi -xu si \\
					`I like someone.' & `I love someone.' \\ \hline
					ga -\N a -ka si & ga -\N a -kaka si \\
					`I am unhappy.' & `I am miserable.' \\ \hline
					\glot a\R i {-g\OO\OO} {\R\OO} & \glot a\R i {-g\OO\OO} {\R\OO\R\OO} \\
					`Please leave.' & `Go away!' \\ \hline
					ni- \glot a\N a -mi si \R i? & ni- \glot a\N a -mi si \R i\R i? \\
					`What is your name?' & `I really need to know your name.' \\
					\hline\hline 
					\end{tabular}
				\caption{One-syllable stem reduplication}
				\end{table}

				\begin{table}[H]
				\centering
					\begin{tabular}{ll}
					Original Form & Reduplicated Form \\ \hline\hline
					gaku & gakugaku \\
					`good' & `very good' \\ \hline
					ga\N a & ga\N aga\N a \\
					`bad' & `very bad' \\ \hline
					\glot a\F a & \glot a\F a\glot a\F a \\
					`Yes.' & `Of course!' \\ \hline
					\glot i\glot i & \glot i\glot i\glot i\glot i \\
					`No.' & `Absolutely not!' \\ \hline
					ka\R i -su & ka\R ika\R i -su \\
					`sleepy' & `exhausted' \\ 
					\hline\hline 
					\end{tabular}
				\caption{Two-syllable stem reduplication}
				\end{table}

		\subsection{Three-syllable stems and up}
		Stems that are three syllables or over undergo opposite-edge reduplication of two syllables. These two syllables are selected right-to-left and then prefixed.
				\begin{table}[H]
				\centering
					\begin{tabular}{ll}
					Original Form & Reduplicated Form \\ \hline\hline
					\glot inaka\R a -su & ka\R a\glot inaka\R a -su \\
					`boring' & `extremely boring' \\
					\hline\hline
					\end{tabular}
				\caption{Three-syllable and up stem reduplication}
				\end{table}

	\section{Pluralization}
		Pluralization is cause by opposite-edge reduplication. One syllable from the beginning of the word is suffixed to denote a plural entity.
			\begin{table}[H]
			\centering
				\begin{tabular}{ll}
				Original Form & Reduplicated Form \\ \hline\hline
				\glot\OO\glot ati & \glot\OO\glot ati\glot\OO \\
				`dog' & `dogs' \\ \hline
				mi\R\OO & mi\R\OO mi \\
				`cat' & `cats' \\ \hline
				\F aa\B u & \F aa\B u\F a \\
				`home' & `homes' \\ \hline
				di & didi \\
				`day' & `days' \\
				\end{tabular}
			\caption{Pluralization}
			\end{table}


\end{document}

