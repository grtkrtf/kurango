\documentclass[12pt]{article}
	%FORMATTING
		\usepackage{graphicx} %op shit
		\usepackage[margin=1in]{geometry} %Set margins
		\usepackage{fixltx2e} %subscript
	%LINGUISTICS-RELATED PACKAGES
		\usepackage{qtree} %syntax trees
		\usepackage{covington} %interlinearizations
		\usepackage{amsmath,amsthm,amssymb} %empty set and some other shit
		\usepackage{tipa} %ipa
		\usepackage{float} %resizing tables

	\let\oldemptyset\emptyset %Redefines the shitty empty set symbol to the old one.
	\let\emptyset\varnothing

	%stuff for handling how awful it is to use default tipa syntax
		\newcommand{\R}{\textipa{R}} %alv tap
		\newcommand{\N}{\textipa{N}} %eng
		\newcommand{\OO}{\textipa{O}} %open o, DONT FORGET THE SECOND O!!!
		\newcommand{\B}{\textipa{B}} %bilab fric voi
		\newcommand{\G}{\textipa{G}} %velar fric voi
		\newcommand{\W}{\textipa{V}} %bilab approximant
		\newcommand{\F}{\textipa{F}} %-voi bilab fric
		\newcommand{\M}{\textturnmrleg} %velar approx
		\newcommand{\glot}{\textipa{P}}
		%\textipa{:}

		\newcommand{\0}{$\emptyset$} %Null
		\newcommand{\D}{\scshape} %smallcaps

		\newcommand{\kurango}{K\textipa{uRaNO}}
		\newcommand{\nom}{\D{nom}}
		\newcommand{\erg}{\D{erg}}
		\newcommand{\abs}{\D{abs}}
		\newcommand{\gen}{\D{gen}}
		\newcommand{\poss}{\D{poss}}
		\newcommand{\emo}{\D{emo}}
		\newcommand{\cop}{\D{cop}}
		\newcommand{\caus}{\D{caus}}
		\newcommand{\pst}{\D{pst}}
		\newcommand{\fut}{\D{fut}}
		\newcommand{\negative}{\D{neg}}
		\newcommand{\interr}{\D{interr}}
		\newcommand{\reas}{\D{reas}}
		\newcommand{\dirprox}{\D{dir:prox}}

%--------------------------------------------------------------%
%  EXAMPLE INTERLINEAR GLOSS, ADAPTED FROM LEV'S DISSERTATION  
%															   
%	\begin{example}											   
%	\label{ex:nasalpoa}										   
%		Ontagake. \textipa{[ontagakse]}						   
%		\gll o= n- tag -ak -e 								   
%		3nmS= irreal burn -perf -irreal.i 					   
%		\glt `It will burn.' 								   
%		\glend 												   
%	\end{example} 											   
%  															   
%	%Reference examples like this: (\ref{ex:nasalpoa})         
%--------------------------------------------------------------%

%This is a test\footnote{This is not a test; it's an example of a footnote.}

\begin{document}
\title{The Phonological Rules of \kurango}
\maketitle

\section{Glide formation}
Syllables ending in /i/ followed by syllables starting in /\glot/ undergo glide formation. The glottal stop is deleted, and the /i/ becomes [j] ([\M] in some dialects). This dictates two crucially ordered phonological rules.
	
	\begin{enumerate}
	\centering
		\item %Glottal stop deletion
			 /\glot/ $\longrightarrow [\emptyset] 
			 /
			 \left[\begin{array}{l} -\text{cons}\\+\text{front}\\+\text{high}\\+\text{syl} \end{array}\right]$ \rule[-10pt]{20pt}{1pt} $\left[\begin{array}{l} -\text{cons}\\+\text{syl} \end{array}\right]$ \\ \emph{Glottal stop deletion}
		\item %Glide formation
			$\left[\begin{array}{l} -\text{cons}\\+\text{front}\\+\text{high}\\+\text{syl} \end{array}\right] \longrightarrow \left[\begin{array}{l} -\text{syl} \end{array}\right]$ / \rule[-10pt]{20pt}{1pt} $\left[\begin{array}{l} -\text{cons}\\+\text{syl} \end{array}\right]$ \\ \emph{Glide formation}
 	\end{enumerate}

 It should be noted that this process does not occur across word boundaries.




\end{document}

