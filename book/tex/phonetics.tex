\chapter{Phonetics}
\label{chap:phone}

\kurango, being an academically-minded language, has an unnatural, engineered phonology tailored for its orthographic system\footnote{For more on the orthography, please refer to \autoref{chap:ortho}.}. However, there is some allophony, not to sell the phonetic system as entirely robotic.

\section{Consonants}
In short, there are three kinds of contrasts in \kurango: voicing, place, and manner. Permutably, there are three places of articulation, four manners of articulation, and two voicing methods. This allows for the existence of 24 phonemes, but, due to typological rarity, voiceless nasals and approximants have been omitted. There is also a glottal stop /\glot/ because glottal stops are awesome.

	\begin{figure}[H]
	\label{permute_cons}
	\begin{enumerate}
	\centering
		\item[Manner:] [+nasal], [-cont, -son], [+cont, -son], [+cont, +son]
		\item[Place:] [+labial], [+coronal], [-coronal]
		\item[Voice:] [+voice], [-voice]
	\end{enumerate}

	\begin{center}
		\caption{Possible values for permutable categories in {\kurango} consonants}
	\end{center}
	\end{figure}

The below consonant chart may help indicate my point. Items in parentheses are used only in allmorphy, items joined with tildes are in free variation.
	
	\begin{table}[H]
	\centering
	\caption{Consonant chart for \kurango}
	\label{conschart}
		\begin{tabular}{ccccc}
		 & Labial & Alveolar & Velar & Glottal \\ \hline\hline
		Nasal & m & n & \N & \\ \hline
		Stop & p b & t d & k g & \glot \\ \hline
		Fricative & {\F} \B & s z & x \G & \\ \hline
		Sonorant & {\W} (w) & \R\tild\textipa{\*r}\tild r &{\M} (j\footnotemark) & \\ \hline\hline
		\end{tabular}
	\end{table}
	\footnotetext{Yes, I know /j/ is palatal, fuck you}

For information on allophony, see \autoref{chap:phono}.

\section{Vowels}
\kurango's vowel system is also ``engineered'' based off permutable contrasts, with height and laterality being contrastive. Roundness is non-contrastive, but front\footnote{Technically, non-back vowels as /a/ is [-front, -back].} vowels are [-round] and back vowels are [+round]. Vowels also contrast in length.

	\begin{figure}[H]
	\label{permute_vow}
	\begin{enumerate}
	\centering
		\item[Height:] [+high], [-high]
		\item[Laterality:] [+back], [-back]
	\end{enumerate}

	\begin{center}
		\caption{Possible values for permutable categories in {\kurango} vowels}
	\end{center}
	\end{figure}

Also, a vowel trapezium:

	\begin{figure}[H]
	\label{trapezium}
	\begin{center}
		\begin{vowel}[three]
			\putcvowel[l]{i}{1}
			\putcvowel[l]{a}{4}
			\putcvowel[r]{u}{8}
			\putcvowel[r]{\textopeno\tild o}{6}
		\end{vowel}
	\caption{Vowel trapezium for \kurango}
	\end{center}
	\end{figure}

There isn't much else to say about vowels, except that /\OO/ is often phonetically lower than it appears in standard IPA, making it sound closer to [\textipa{6}]. L1 bias may cause some {\kurango} vowels to phonetically shift: for me (a Californian), /\OO/ is closer to [o], /u/ is closer to [\textipa{1}], and /a/ is centralized.