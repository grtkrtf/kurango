\section{Case assignment}
	In \kurango , case assignment is based on word order. Arguments to the get ergative case, and arguments to the right of the verb get absolutive case.

	\begin{example}
	\label{ex:case_wordorder_1}
		Ni'inakarana. [\textipa{niPinakaRana}]
		\gll ni- inakara -na
		2- bore -1
		\glt `You bore me.'
		\glend
	\end{example}

	\begin{example}
	\label{ex:case_wordorder_2}
		Na'inakarani. [\textipa{naPinakaRani}]
		\gll na- inakara -ni
		1- bore -2
		\glt `I bore you.'
		\glend
	\end{example}

	This ordering principle also allows us to categorize {\kurango} as a Fluid-S language, as arguments of intransitive verbs (S) pattern either with subjects of transitive verbs (S\textsubscript{A}) or objects of transitive verbs (S\textsubscript{O}) based on their position relative to the verb. Determining the way in which S patterns has to do with the semantics of the utterance: S\textsubscript{A} (marked with {\D{erg}}) has no entailment about the volition of the Agent in the action being performed, while S\textsubscript{O} (marked with {\D{abs}}) entails that the agent was volitional in the action being performed.

	\begin{example}
	\label{ex:case_erg}
		Nakari. [naka\R i]
		\gll na- kari
		1- sleep
		\glt `I sleep.' (by my own volition.)
		\glend
	\end{example}

	\begin{example}
	\label{ex:case_abs}
		Karina. [ka\R ina]
		\gll kari -na
		sleep -1
		\glt `I sleep.' (no entailment about my volition.)
		\glend
	\end{example}

	%further analysis here