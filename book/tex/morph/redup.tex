\section{Reduplication}

	One of \kurango 's hallmarks is its reduplicative system. It has multiple reduplicative processes.

	\subsection{Intensification}
	\label{redup_intens}
		 Semantically, this reduplicative process results in an intensification of the meaning of the morpheme. This more or less restricts where reduplication can or can not occur---it is common on intensifiable parts of speech like adjectives, and very rare on less-intensifiable parts of speech like verbs. The reduplicative template varies based on how many syllables\footnote{Syllable being one vowel and its onset, or one grapheme in the \emph{kuraito} orthography.} are in the stem.

		\subsubsection{One and two-syllable stems}
			One and two-syllable stems are copied and reduplicated entirely\footnote{Forms in these tables ignore phonological rules and represent underlying forms.}.
				\begin{table}[H]
				\centering
					\begin{tabular}{ll}
					Original Form & Reduplicated Form \\ \hline\hline
					ga -ti -xu si & ga -titi -xu si \\
					`I like someone.' & `I love someone.' \\ \hline
					ga -\N a -ka si & ga -\N a -kaka si \\
					`I am unhappy.' & `I am miserable.' \\ \hline
					\glot a\R i {-g\OO\OO} {\R\OO} & \glot a\R i {-g\OO\OO} {\R\OO\R\OO} \\
					`Please leave.' & `Go away!' \\ \hline
					ni- \glot a\N a -mi si \R i? & ni- \glot a\N a -mi si \R i\R i? \\
					`What is your name?' & `I really need to know your name.' \\
					\hline\hline 
					\end{tabular}
				\caption{One-syllable stem reduplication}
				\end{table}

				\begin{table}[H]
				\centering
					\begin{tabular}{ll}
					Original Form & Reduplicated Form \\ \hline\hline
					gaku & gakugaku \\
					`good' & `very good' \\ \hline
					ga\N a & ga\N aga\N a \\
					`bad' & `very bad' \\ \hline
					\glot a\F a & \glot a\F a\glot a\F a \\
					`Yes.' & `Of course!' \\ \hline
					\glot i\glot i & \glot i\glot i\glot i\glot i \\
					`No.' & `Absolutely not!' \\ \hline
					ka\R i -su & ka\R ika\R i -su \\
					`sleepy' & `exhausted' \\ 
					\hline\hline 
					\end{tabular}
				\caption{Two-syllable stem reduplication}
				\end{table}

		\subsubsection{Three-syllable stems and up}
		Stems that are three syllables or over undergo opposite-edge reduplication of two syllables. These two syllables are selected right-to-left and then prefixed.
				\begin{table}[H]
				\centering
					\begin{tabular}{ll}
					Original Form & Reduplicated Form \\ \hline\hline
					\glot inaka\R a -su & ka\R a\glot inaka\R a -su \\
					`boring' & `extremely boring' \\
					\hline\hline
					\end{tabular}
				\caption{Three-syllable and up stem reduplication}
				\end{table}

	\subsection{Pluralization}
		Pluralization is cause by opposite-edge reduplication. One syllable from the beginning of the word is suffixed to denote a plural entity. 
			\begin{table}[H]
			\centering
				\begin{tabular}{ll}
				Original Form & Reduplicated Form \\ \hline\hline
				\glot\OO\glot ati & \glot\OO\glot ati\glot\OO \\
				`dog' & `dogs' \\ \hline
				mi\R\OO & mi\R\OO mi \\
				`cat' & `cats' \\ \hline
				\F aa\B u & \F aa\B u\F a \\
				`home' & `homes' \\ \hline
				\end{tabular}
			\caption{Pluralization}
			\end{table}
		Monosyllabic words are pluralized using the suffix \emph{-ti} instead. This suffix probably arose from confusion regarding plural person-markers and plural pronouns both being \emph{nana, nini} and \emph{nunu}.

