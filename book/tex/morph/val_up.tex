	\section{Valence-increasing morphology}
		\subsection{Applicatives}
			Applicatives in {\kurango} are marked using a suffix on the introduced argument, -\textipa{P}uti.
			\begin{example}
			\label{ex:nonapplicative}
				Nimotivagaana. \textipa{[nimOtiBaga:na]}
				\gll ni- {\0-} m\OO ti -\B a -ga\textipa{:} -na -\0
				2- {\erg-} die -{\caus} -{\fut} -1 -\abs
				\glt `You are going to kill me.'
				\glend
			\end{example}

			\begin{example}
			\label{ex:applicative}
				Nimotivagaana soodongu'uti. \textipa{[nigu naku mOtiBaga: sO:dONuPuti]}
				\gll ni- {\0-} m\OO ti -\B a -ga\textipa{:} -na -\0 {s\OO\textipa{:}d\OO} -\N u -\textipa{P}uti
				1- {\erg}- kill -{\caus} -{\fut} -1 -{\abs} sword {-\gen} -{\D{appl}}
				\glt `You are going to kill me with a sword.' (lit: `You made me die with a sword.')
				\glend
			\end{example}

		\subsection{Causatives}
			Causatives in {\kurango} are marked using the verbal suffix -\B a.

			\begin{example}
			\label{ex:non_causative}
				Nakumotigaa. \textipa{[nakumOtiga:]}
				\gll na- ku- m\OO ti -ga\textipa{:}
				{\D{1}}- {\D{nom}}- die -{\D{fut}}
				\glt `I am going to die.'
				\glend
			\end{example}

			\begin{example}
			\label{ex:causative}
				Namotivagaana. \textipa{[namOtiBaga:na]}
				\gll na- {\0-} m\OO ti -\B a -ga\textipa{:} -na -\0
				{\D{1}} {\D{erg}}- die -{\D{caus}} -{\D{fut}} -1 -\abs
				\glt `I am going to kill myself.' (lit: `I am going to make myself die.')
				\glend
			\end{example}

		If an agentive argument is not introduced with the causative suffix -\B a, the utterance is still grammatical, but it has a passivized 
		%what should i call this??
		 connotation to it.

		 	\begin{example}
		 	\label{ex:noncausative}
		 		Nakumotivagaa. \textipa{[nakumOtiBaga:]}
		 		\gll na- ku- m\OO ti -\B a -ga\textipa{:}
		 		1- {\nom-} die {-\caus} {-\fut}
		 		\glt `I am going to be killed' (lit: I am going to be made dead) %Is this future perfect or just future? wtf
		 		\glend
		 	\end{example}