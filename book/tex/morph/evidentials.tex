\section{Evidentiality-encoding morphology}

	%kixada = evidential adverb
		%-ngo sound, language
		%-riiti sight
		% -numungu smell
		% vafawa taste
		% tuusi touch
			% ngatuusi pain (neg touch)
				% Compared to gangapoka, which is like the perceived pain, ngatuusi has a more physiological meaning
		% miiti'a sense (holistic)
			% miiti'aa sensation

	% kura think
	% rhana know
	% -kaja real
	% -nga irreal


\subsection{Discussion of knowledge}
	Knowledge is a very important concept for {\kurango} speakers, and the topic has more depth than the English language provides.

	\emph{\G a} serves as the root noun for ``general knowledge,'' but there are many semantically-related words that refer to different types of knowledge. These lexicalized words for different types of knowledge reflects what types of knowledge {\kurango} speakers believe to be important.
		\begin{enumerate}
			\item \emph{\G a\N aka\M a} is knowledge that is false or otherwise irrelevant to the current discourse.
			\item \emph{\G amuzu} is introspective knowledge, correlating somewhat to a ``sense of self.''
			\item \emph{\G as\OO t\OO\R ii} refers to social knowledge.
			\item \emph{\G apapi'a} refers to fact-based knowledge. It does not have to be applicable in any pragmatic context.
			\item \emph{\G ana\G aa} refers to knowledge in general, sort of like ``world smarts.'' It is regarded as a high compliment and the most difficult type of knowledge to obtain.
		\end{enumerate}

% talk about how you know something
% different types of knowledge shoehorned in here too!!!!!