\section{Numeracy}

	\subsection{The cardinal root \emph{\R\OO}}
		Cardinal numbers in {\kurango} are marked with the cardinal root \emph{\R\OO}. The vowel can be lengthened to form \emph{\R\OO\OO}, ``number'' as an abstract concept. However, it more commonly appears with the numeracy suffixes, discussed below.
	\subsection{Numeracy suffixes}
		\kurango 's capacity to encode any number comes from the combination of ten morphemes that represent the numbers 0-9. These morphemes have underlying forms that obey syllable structure rules for \kurango, but they are commonly discussed as though they only have consonants in their stems. This is because of an overarching vowel harmony rule based on whether or not the initial numeral suffix is odd or even (see below). 
		% couldn't get autoref to properly work here
			\begin{table}[H]
			\centering
			\caption{{\kurango} numeracy suffixes}
			\label{numeracy_table}
				\begin{tabular}{c|ccc}
					Numeral & Suffix & -a glyph & -u glyph \\ \hline\hline
					1 & -xa\R a & 
						\includegraphics[scale=0.25]{././img/1A.png} &
						\includegraphics[scale=0.25]{././img/1U.png} \\
					2 & -duzu &
						\includegraphics[scale=0.25]{././img/2A.png} &
						\includegraphics[scale=0.25]{././img/2U.png} \\
					3 & -ta\R a &
						\includegraphics[scale=0.25]{././img/3A.png} &
						\includegraphics[scale=0.25]{././img/3U.png} \\
					4 & -kutu &
						\includegraphics[scale=0.25]{././img/4A.png} &
						\includegraphics[scale=0.25]{././img/4U.png} \\ 
					5 & -sa\N a &
						\includegraphics[scale=0.25]{././img/5A.png} &
						\includegraphics[scale=0.25]{././img/5U.png} \\
					6 & -zusu &
						\includegraphics[scale=0.25]{././img/6A.png} &
						\includegraphics[scale=0.25]{././img/6U.png} \\
					7 & -sata &
						\includegraphics[scale=0.25]{././img/7A.png} &
						\includegraphics[scale=0.25]{././img/7U.png} \\
					8 & -\glot u\glot u &
						\includegraphics[scale=0.25]{././img/8A.png} &
						\includegraphics[scale=0.25]{././img/8U.png} \\
					9 & -na\B a &
						\includegraphics[scale=0.25]{././img/9A.png} &
						\includegraphics[scale=0.25]{././img/9U.png} \\
					0 & -zu\R u &
						\includegraphics[scale=0.25]{././img/0A.png} &
						\includegraphics[scale=0.25]{././img/0U.png} \\ \hline
				\end{tabular}
			\end{table}

		So, \emph{roxara} would gloss to ``one,'' but these suffixes can also attach to nouns. An example would be \emph{o'atikutu}, or "four dogs."

		\subsubsection{Larger numbers and vowel harmony}
		\label{number_harmony}
			A language should, naturally, have a way to discuss really large numbers. In \kurango, this is as simple as attaching multiple suffixes to \emph{\R\OO}. However, there is a catch:

				\begin{figure}[H]
				\label{interlin_num_harm}
					\begin{example}
					\label{num_a_harm}	
						roxarazarasata [\R\OO.\stress xa.\R a.za.\R a.sa.ta]
						\gll {\R\OO} -xa\R a -zu\R u -sata
						\D{num} -one -zero -seven
						\glt `one hundred seven'
						\glend
					\end{example}

					\begin{example}
					\label{num_u_harm}
						*roxarazurusata [\R\OO.\stress xa.\R a.zu.\R u.sa.ta]
						\gll {\R\OO} -xa\R a -zu\R u -sata
						\D{num} -one -zero -seven
						\glt `(Intended) one hundred seven'
						\glend
					\end{example}
				\end{figure}

			As stated earlier, there is an overarching vowel harmony rule based on whether or not the initial numeral suffix is odd or even. Looking at the numeracy suffixes above, a noticeable pattern emerges, where odd numeral suffixes have /a/ as their vowels and even numeral suffixes (and zero) have /u/ as their vowels. The suffix that attaches closest to the root \emph{\R\OO} determines the vowel harmony for the rest of the numeral construction.

	\subsection{Ordinality}
		Ordinality in {\kurango} is a lot more simple than English ordinality. There are no semantic classes of objects that take different ordinal morphology (take ``tertiary'' versus ``third'')---all take the same derivational suffix \emph{-ta}. For example, \emph{o'atitakutu} would translate to ``the fourth dog'' (compare with \emph{o'atikutu} ``four dogs'' above).

		\subsubsection{Ordinality and calendar terms}
			The ordinal suffix is used on the roots ``year,'' ``day,'' ``week,'' and ``month'' to express calendar terms.

			\begin{itemize}
			\item \emph{di} is the root for ``day.'' \emph{ditaxa\R ana\B a} would translate to ``the nineteenth day,'' and \emph{dixa\R ana\B a} to ``nineteen days.''

			\item \emph{\glot u} is the root for ``week.'' \emph{\glot uta\glot u\glot uzu\R u} would translate to ``the eightieth week,'' and \emph{\glot u\glot u\glot uzu\R u} to ``eighty weeks.''

			\item \emph{m\OO} is the root for ``month.'' \emph{m\OO tata\R a} would translate to ``March (lit: the third month),'' and \emph{m\OO ta\R a} to ``three months.''

			\item \M i is the root for ``year.'' \emph{\M itaduzuzu\R uxu\R usu\N u} would translate to ``2015 (lit: the 2015th year),'' and \emph{\M iduzuzu\R uxu\R usu\N u} to ``2015 years.''
			\end{itemize}

	\subsection{Multiplicative adverbs}
		Cardinal number constructions can be derived into multiplicative adverbs ``double, triple, quadruple'' as they can in English, using an additional root that is mutually exclusive with the noun slot. This root is \emph{d\OO\R\OO}. An example would be \emph{d\OO\R\OO sata}, which translates to ``septuple.''

	\subsection{Morphosyntax}
		\begin{table}[H]
		\centering
		\label{num_morphosyntax}
			\begin{tabular}{cccc}
			Root & Slot 1 & Slot 2 & Slot 3-$\infty$ \\ \hline\hline
			\makecell{\R\OO\\d\OO\R\OO\\any noun} & \makecell{-ti\\
			%-d\OO THIS WAS IN THIS SLOT ALSO, WHAT DOES IT MEAN
			} & \makecell{Numeracy suffix\\(that determines vowel harmony)} & \makecell{Numeracy suffix\\(that undergoes vowel harmony)} \\ \hline
			\end{tabular}
			\caption{Morphosyntax of numeracy constructions}
		\end{table}

	\subsection{Other useful numeracy constructions}
		There are several other common constructions where numeracy suffixes come in handy.

		\begin{itemize}
			\item Last names: \emph{\glot a\N ataduzu} ``last name (lit: second name)''
			\item Secondary emotions: \emph{gatitixu si ga\N a\W u\F upataduzu si} ``I love someone, but I am also scared (lit: I love someone; my second emotion is fear)'' % May not be grammatically correct. Possibly missing something like ``but''
		\end{itemize}