%\entry{Ortho}{IPA}{Partofspeech}{Definition}

%Weird glitches involving columns. look into the package and if that wont do it then do some janky shit to it.

\chapter{Dictionary}
\label{chap:dict}

\section*{A}
\begin{multicols}{2}
	\entry{adara}{\glot a.\stress da.\R a}{verb}{to order. Borrowed from English name ``Adele.''}

	\entry{afa}{\stress\glot a.\F a}{interjection}{yes. Designed to be maximally phonetically distinct from \emph{i'i}, ``no.''}

	\entry{a'i}{\stress\glot a.ji}{noun, determiner}{space, location. As a determiner functions similarly to English ``where.''}

	\entry{a'iramoo}{a.\stress ji.\R a.m\OO\len}{noun}{outsider, alien, outgroup member (pejorative)}

	\entry{anga}{\stress\glot a.\N a}{noun}{name}

	\entry{ari}{\stress\glot a.\R i}{verb}{to go}
\end{multicols}

\section*{B}
\begin{multicols}{2}
	%\entry{Ortho}{IPA}{Partofspeech}{Definition}
	\entry{bako}{\stress ba.k\OO}{noun}{body. Often compounds to form different parts of the body, e.g. \emph{vafawaabako} ``mouth.''}

	\entry{bo'i}{bji}{determiner}{how, by}

	\entry{boraa}{b\OO.\stress\R a\len}{inf. morpheme}{progressive aspect marker. Historically served as a present tense marker but that is now null.}
\end{multicols}

\section*{D}
\begin{multicols}{2}
	\entry{-da}{da}{infl. morpheme}{emotion marker for long duration}

	\entry{di}{di}{noun}{day. Borrowed from English ``day.''}

	\entry{doro}{\stress d\OO.\R\OO}{adverb}{multiplicative adverb. Compare to English ``-tuple'' morph.}

	\entry{duru}{\stress du.\R u}{verb}{to follow. Borrowed from English name ``Dora.''}

	\entry{duzu}{\stress du.zu}{quantifier}{two. Borrowed loosely from French \emph{deux}, ``two.''}

\end{multicols}

\section*{F}
\begin{multicols}{2}
	\entry{-fa}{fa}{infl.morpheme}{marks a target as proximal} % IS THIS A PREPOSITION?

	\entry{faavu}{\stress\F a\len.\B u}{noun}{home}

	\entry{fivu}{\stress\F i.\B u}{noun}{house}

	\entry{-fu}{\F u}{infl. morpheme, pronoun}{marks an inanimate target.  Also serves as the emotion marker for an inanimate target.}
\end{multicols}

\section*{G}
\begin{multicols}{2}
	\entry{ga}{ga}{adjective}{emotional}

	\entry{-gaa}{ga\len}{infl. morpheme}{future tense marker}

	\entry{gaaro}{\stress ga\len.\R\OO}{verb}{select}

	\entry{gakagangasiri}{\stress ga.ka.\stress ga.\N a.\stress si.\R i}{interjection}{lexicalized expression means ``How are you?'' Literally translates to ``*Good emotion, bad emotion, is it?''}

	\entry{gaka}{\stress ga.ka}{adjective}{lexicalized expression that means ``good.'' Literally means ``good emotion.''}

	\entry{gakasi}{ga.\stress ka.si}{interjection}{lexicalized expression that means ``Hello.'' Literally translates to ``*Emotion is good.''}

	\entry{ganga}{\stress ga.\N a}{adjective}{lexicalized expression that means ``bad.'' Literally translates to ``*negated emotion.''}

	\entry{gangasi}{ga.\stress\N a.si}{interjection}{lexicalized expression that means ``Hello.'' Literally translates to ``*Emotion is negated.''}

	\entry{go'i}{\stress g\OO .\glot i}{verb}{to desire, want}

	\entry{goo}{g\OO\len}{preposition}{ablative preposition; ``from''}

	\entry{gokoo}{\stress g\OO.k\OO\len}{preposition}{illative preposition; ``into, back from''}

	\entry{gooko}{\stress g\OO\len.k\OO}{preposition}{departative preposition; ``leave (with intention of returning).''}

\end{multicols}

\section*{I}
\begin{multicols}{2}
	\entry{i'a}{\stress i.ja}{preposition}{oblique preposition; ``east of''}

	\entry{i'i}{\stress\glot i.ji}{interjection}{no. Designed to be maximally phonetically distinct from \emph{afa}, ``yes.''}

	\entry{inakara}{\glot i.\stress\N a.ka.\R a}{verb}{to bore. Borrowed from English surname ``Inkelas.''}
\end{multicols}

\section*{J}
\begin{multicols}{2}
	\entry{-ja}{\M a}{deriv. morpheme}{serves as a person marker (example: ``learn'' to ``learner''). Changes word class from verb to noun.}

	\entry{ji}{\M i}{noun}{year. Borrowed loosely from English ``year.''}

	\entry{jivi}{\stress\M i.\B i}{verb}{give. Borrowed loosely from English ``give.''}

\end{multicols}

\section*{K}
\begin{multicols}{2}
	\entry{-ka}{ka}{deriv. morpheme}{emotion marker for ``happiness.'' Changes word class from noun to adjective.}

	\entry{kanga}{\stress ka.\N a}{infl. morpheme}{perfective marker. Combines with future and past tense markers to form past perfect and future perfect.}

	\entry{kari}{\stress ka.\R i}{verb}{to sleep. Truncated form of Japanese name \emph{\begin{CJK}{UTF8}{min}紫\end{CJK}} ``Yukari.''}

	\entry{karipa}{ka.\stress\R i.pa}{verb, noun}{to nap, a nap. Exocentric compound of \emph{kari} ``to sleep'' and \emph{pa}, the short emotion durative.}

	\entry{kii}{ki\len}{noun, determiner}{time. As a determiner, functions similarly to English ``when.''}

	\entry{kogoo}{\stress k\OO.g\OO\len}{preposition}{elative preposition; ``from, out of''}

	\entry{koo}{k\OO\len}{preposition}{allative preposition; ``to''}

	\entry{koogo}{\stress k\OO\len.g\OO}{preposition}{redepartitive preposition, ``come with intention of leaving''}

	\entry{koogoo}{\stress k\OO\len .g\OO\len}{preposition}{preposition that means ``to and from.'' An echo construction based on prepositions \emph{koo} ``to'' and \emph{goo} ``from.''} % get a better gloss for this...

	\entry{kopanga}{k\OO.\stress pa.\N a}{noun}{friend, borrowed from French \emph{copain}}

	\entry{ku'a}{kwa}{noun}{time}

	\entry{kura}{\stress ku.\R a}{noun}{brain}

	\entry{kura}{\stress ku.\R a}{verb}{to think}

	\entry{kura'ito}{ku.\stress\R a.ji.t\OO}{noun}{writing} % make sure u add more info on the etymology of this later as it IS a compound word.

	\entry{kuraja}{ku.\stress\R a.\W a}{noun}{person. Excocentric compound of ``think'' and person marker---literally ``thinker.''}

	\entry{kurango}{ku.\stress\R a.\N\OO}{noun}{language} % make sure u add more info on the etymology of this later as it IS a compound word.

	\entry{kutu}{\stress ku.tu}{quantifier}{four. Borrowed loosely from French \emph{quatre}, ``four.''}
\end{multicols}

\section*{M}
\begin{multicols}{2}
	\entry{-mi}{mi}{determiner}{posessive pronoun marker. Borrowed from English ``my.'' Assigns genitive case.}

	\entry{miro}{\stress mi.\R\OO}{noun}{cat. Borrowed from English name ``Milo.''}

	\entry{-miiti'a}{\stress mi\len.tja}{deriv. morpheme}{sensation suffix for sense (as in, a sixth, holistic sense). Last vowel can be lengthened to mean ``sensation.'' Commonly affixed to the evidential clitic \emph{kixada}.}
	
	\entry{mo}{m\OO}{noun}{month. Borrowed loosely from English ``month.''}

	\entry{-mo}{m\OO}{infl. morpheme}{past tense marker}

	\entry{moti}{\stress m\OO .ti}{verb}{to die. Borrowed loosely from French \emph{mourir} ``die.''}

	\entry{-mu}{mu}{infl. morpheme}{emotion marker for a reflexive target}

	\entry{-muzu}{\stress mu.zu}{deriv. morpheme, pronoun}{reflexive pronoun. Lowers valency of verb.}
\end{multicols}

\section*{N}
\begin{multicols}{2}
	\entry{na}{na}{determiner}{first person marker}

	\entry{na'isaguu}{noun}{na.\stress ji.sa.gu\len}{dog food (ketchup)}

	\entry{nana}{\stress na.na}{determiner}{first person pronoun. Reduplcation of \emph{na}, the first person marker.}

	\entry{nava}{\stress na.\B a}{quantifier}{nine. Borrowed loosely from French \emph{neuf}, ``nine.''}

	\entry{ni}{ni}{determiner}{second person marker}

	\entry{nini}{\stress ni.ni}{determiner}{second person pronoun. Reduplication of \emph{ni}, the second person marker.}

	\entry{no'a}{\stress n\OO.ja}{preposition}{oblique preposition; ``north of''}

	\entry{-no'i}{\stress n\OO.ji}{deriv. morpheme}{emotion marker for a neutered emotion}

	\entry{noriga}{n\O.\stress\R i.ga}{noun}{Norway. Borrowing of Norwegian \emph{Norge}.}

	\entry{nu}{nu}{determiner}{third person marker}

	\entry{-numungu}{nu.\stress mu.\N u}{deriv. morpheme}{sensation suffix for smell. Last vowel can be lengthened to mean ``smell.'' Commonly affixed to the evidential clitic \emph{kixada}. Etymologically derived from association of nasal consonants and smell.}

	\entry{nunu}{\stress nu.nu}{determiner}{third person pronoun. Reduplication of \emph{nu}, the third person marker.}
\end{multicols}

\section*{NG}
\begin{multicols}{2}
	\entry{nga}{\N a}{deriv. morpheme}{negation morpheme}

	%\entry{ngu}{\N u}{infl. morpheme, pronoun}{genitive marker} %DO i even still use this???

	\entry{-nga'uva}{\N a.\stress ju.\B a}{deriv. morpheme}{changes word class from verb to noun} % Is this even ``legal?'' What else should this do? Semantically related to -nga and -va

	\entry{-ngo}{\N\OO}{deriv. morpheme}{sensation suffix for sound, language. Last vowel can be lengthened to mean ``sound.'' Commonly affixed to the evidential clitic \emph{kixada}.}

\end{multicols}

\section*{O}
\begin{multicols}{2}
	\entry{o'ati}{\textipa{PO.\stress wa.ti}}{noun}{dog. Borrowed loosely from English name ``Archie.''}

	\entry{ora}{\stress\glot\OO.\R a}{verb}{to punch. Borrowing of Japanese onomatopoeia \emph{\begin{CJK}{UTF8}{min}オラ\end{CJK}}.}
\end{multicols}

\section*{P}
\begin{multicols}{2}
	\entry{-pa}{pa}{infl. morpheme}{emotion marker for short duration}

	\entry{pada}{\stress pa.da}{noun}{length. Exocentric compound of \emph{pa} and \emph{da}, the emotion duratives.}

	\entry{papi'a}{pa.\stress pja}{noun}{paper. Borrowing of English ``paper.''}

	\entry{-po}{p\OO}{deriv. morpheme}{emotion marker for a ``primal'' emotion} % Explain the semantics? Reference the emotion section?
\end{multicols}

\section*{R}
\begin{multicols}{2}
	\entry{ra'isaa}{\R a.\stress ji.sa\len}{adjective}{superfluous} % What is the etymology of this?

	\entry{ri}{\R i}{particle}{interrogative statement marker}

	\entry{-riiti}{\stress\R i\len.ti}{deriv. morpheme}{sensation suffix for sight. Last vowel can be lengthened to mean ``sight.'' Commonly affixed to the evidential clitic \emph{kixada}. Borrowed from English ``light.''}

	\entry{ro}{\R\OO}{particle}{imperative statement marker}

	\entry{ro-}{\R\OO}{quantifier}{numeracy marker. Serves as the beginnign morpheme for a cardinal number.}

	\entry{roodo}{\stress\R\OO\len.d\OO}{noun}{road. Borrowed from English ``road.''}

	\entry{-ru}{\R u}{infl. morpheme}{emotion marker for a reciprocal target}

	\entry{-ruzu}{\stress\R u.zu}{deriv. morpheme}{reciprocal morpheme. Lowers verb valency.}
\end{multicols}

\section*{RH}
\begin{multicols}{2}
	\entry{rhamuzu}{\G a.\stress mu.zu}{noun}{introspective knowledge. Lexicalization of \emph{\G a} morpheme and the reflexive \emph{-muzu}.}

	\entry{rhana}{\stress\G a.na}{verb}{to know. Used as a general expression.}

	\entry{rhangakaja}{\G a.\stress\N a.ka.\M a}{noun}{a type of knowledge used specifically in instances where the knowledge is irrelevant to the discourse. Loose borrowing of Japanese name \emph{\begin{CJK}{UTF8}{min}羽川\end{CJK}}``Hanekawa''.}

	\entry{rhanarhaa}{\G a.\stress na.\G a\len}{noun}{a type of worldly, difficult-to-obtain knowledge; a high compliment. Etymologically derived from reanalyzed reduplication of the \emph{\G a} cran-morph.}

	\entry{rhapapi'a}{\G a.\stress pa.pja}{noun}{fact-based knowledge. Lexicalization of \emph{\G a} morpheme and \emph{papi'a} ``paper.''}

	\entry{rhasotorii}{\G a.\stress s\OO.t\OO.\R i\len}{noun}{social knowledge. Inspired by English ``street smarts.''}

	\entry{-rhu}{-\G u}{determiner}{dative case marker. Used on indirect objects in ditransitive sentences.}

\end{multicols}

\section*{S}
\begin{multicols}{2}
	\entry{sa'a}{\stress sa.ja}{preposition}{oblique preposition; ``south of''}

	\entry{sanga}{\stress sa.\N a}{quantifier}{five. Loosely borrowed from French \emph{cinq}, ``five.''}

	\entry{saru}{\stress sa.\R u}{verb}{to be able to do something. A modal similar to English ``can.''}

	\entry{sata}{\stress sa.ta}{quantifier}{seven. Loosely borrowed from French \emph{sept}, ``seven.''}

	\entry{si}{si}{verb}{copula. Similar in meaning to English ``to be.''}

	\entry{soodo}{\stress s\OO\OO .d\OO}{noun}{sword. Borrowed from English ``sword.''}

	\entry{-so}{s\OO}{deriv. morpheme}{changes word class from noun to adjective}

	\entry{sotorii}{s\OO.\stress t\OO.\R i\len}{noun}{street. Borrowing of English ``street.''}

	\entry{-su}{su}{deriv. morpheme}{changes word class from verb to adjective}

	\entry{susu}{\stress su.su}{quantifier}{six. Loosely borrowed from French \emph{six}, ``six.''}

	\entry{sutaku}{su.\stress ta.ku}{verb}{to posess multiple earlobes. Loosely borrowed from English surname ``Stockdale.''}

\end{multicols}

\section*{T}
\begin{multicols}{2}
	\entry{-ta}{ta}{deriv. morpheme}{ordinality marker}

	\entry{tara}{\stress ta.\R a}{quantifier}{three. Loosely borrowed from French \emph{trois}, ``three.''}

	\entry{-ti}{ti}{deriv. morpheme}{emotion marker for ``affection.'' Changes word class from noun to adjective.}

	% Theres also a -ti plural morph, but may be obsoleted by opposite edge reduplication process. Maybe only for monosyllabic roots?

	\entry{-to}{t\OO}{deriv. morpheme}{emotion marker for ``excitement.'' Changes word class from noun to adjective.}

	\entry{-tuusi}{\stress tu\len.si}{deriv. morpheme}{sensation suffix for touch. Last vowel can be lengthened to mean ``touch.'' Commonly affixed to the evidential clitic \emph{kixada}. Loosely borrowed from English ``touch.'' Can be negated to mean ``pain'' (\emph{ngatuusii}).}
\end{multicols}

\section*{U}
\begin{multicols}{2}
	\entry{u}{\glot u}{noun}{week. Borrowed loosely from English ``week.''}

	\entry{uba'a}{\glot u.\stress bja}{preposotion}{oblique preposition; ``over, above''}

	\entry{uda'a}{\glot u.\stress dja}{preposition}{oblique preposition; ``under, below''}

	\entry{usu}{\stress\glot u.su}{adverb}{also}

	\entry{uti}{\stress\glot u.ti}{deriv. morpheme}{applicative morpheme. Increases verb valency. If attached to a noun, changes class to a verb.}

	\entry{u'u}{\stress\glot u.wu}{quantifier}{eight. Borrowed loosely from French \emph{huit}, ``eight.''}
\end{multicols}

\section*{V}
\begin{multicols}{2}
	\entry{-va}{\B a}{deriv. morpheme}{causative morpheme. Increases verb valency.}

	\entry{-vafawa}{\B a.\stress\F a.\W a}{deriv. morpheme}{sensation suffix for taste. Last vowel can be lengthened to mean ``taste.'' Commonly affixed to the evidential clitic \emph{kixada}. Etymotically derived from the associaton of labial consonants and taste.}

	\entry{voru}{\stress\B\OO.\R u}{verb}{please (as in ``to please someone.'' Borrowed from the English name ``Vore.'')}
\end{multicols}

\section*{W}
\begin{multicols}{2}
	\entry{wa}{\W a}{determiner}{that, this (depending on attached morphology)}

	\entry{wa'a}{\stress\W a.ja}{preposition}{oblique preposition; ``west of''}

	\entry{wooka}{\stress\W\OO\len .ka}{deriv. morpheme, determiner}{reason clitic. Marks a subordinate clause that gives a reason for another clause. As a determiner, functions similarly to English ``why.''}

	\entry{-wu}{\W u}{deriv. morpheme}{emotion marker for ``confidence.'' Changes word class from noun to adjective.}
\end{multicols}

\section*{X}
\begin{multicols}{2}
	\entry{-xa}{xa}{deriv. morpheme}{emotion marker for ``pride.'' Changes word class from noun to adjective.}

	\entry{xara}{\stress xa.\R a}{quantifier}{one}

	\entry{xoranu}{x\OO.\stress\R a.nu}{verb}{to sweat. Borrowed loosely from English surname ``Holland.''}

	\entry{-xu}{xu}{infl. morpheme}{marker for an animate target. Also serves as the emotion marker for an animate target.}
\end{multicols}

\section*{Z}
\begin{multicols}{2}
	\entry{-za}{za}{infl. morpheme}{marks a target as distal}

	\entry{-zo}{z\OO}{deriv. morpheme}{changes word class from noun to adverb}

	\entry{-zu}{zu}{deriv. morpheme}{changes word class from verb to adverb}

	\entry{zuru}{\stress zu.\R u}{quantifier}{zero. Borrowed loosely from French \emph{zero}, ``zero.''}

\end{multicols}


 