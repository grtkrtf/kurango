\section*{To-do}

\emph{This section will be omitted in the final compile. It's so Garret knows what to work on.}\\

Sections to write from notebook:
	\begin{enumerate}
		\item Lexically different types of intelligence.
		% Kura -tifi = something good as an easter egg
		\item case assignment
		\item add \emph{rha} as itself for a noun that means thought. Makes a lot more sense for your compounding ideas that you were meditating upon.
	\end{enumerate}

The -ti/opposite edge reduplication issue is pretty pressing. for the sake of continuity i prefer to use OER but the nana/nini/nunu thing might be too salient to use that. think of some historically motivated solution? the current one is kind of weak.

Old stuff to consolidate:
	\begin{enumerate}
		\item check interlins
		\\ Maybe like print out a copy and make sure everything is up to date (physical proofreading ftw)(for the interlins only.)
		\item move unfinished ``philosophy'' section to the introduction instead?
			\\ finish it too lol
	\end{enumerate}

New stuff to write:
	\begin{enumerate}
		\item Think of a system of encoding grammatical mood. This is necessary in order to complete your own litmus test, a translation of the ``Litany Against Fear.'' Moods that already have a gloss: Indicative(\0), Imperative(\R\OO), Interrogative(\R i); Moods that need a gloss: Conditional ``would'', Subjunctive ``if'', Potential ``may''
		\item Think of ways to incorporate zero-derivation that won't fuck with the use of your causative. Trigger: salience, close semantic proximity, etc. (god forbid something is COMPLETELY irregular)
		\item Culture/sociolinguistics/dialectology section? (Talk about the progressive--descriptive vs conservative--prescriptive dialects.)
		\item WH QUESTIONS!!! look at discussion with jenks. u need a new lexical entry for this.
		\item Where are you going to put your morphosyntax section? (A: probably in with the morphology as you introduce concepts.)
		\item section on adverbial constructions
		\item Different morphological topics
		\\ Like, prox/dist/near/far distinction (Should probably go in case section)
		\item Inflectional morphology (how 2 organize?)
		\item Derivational morphology (how 2 organize?)
		\\ 1 section on the dedicated class changers
		\\ 1 section on everything else? % Do you even need these sections?
		\item word order section (under grammar)
		\item a ``fun facts'' appendix or some shit?? maybe put in intro instead?
		\item Suprasegmental stuff. (tone, stress, others)
		\item Some phonology about vowel deletion in common grammatical forms (person markers, tense markers, etc) to form nasal consonant clusters? u need those nasals bb
		\item eventually find some way to make numeracy not as lame.
	\end{enumerate}

Stuff to yoink from the wiki:
	\begin{enumerate}
		\item person marking/(Case too?) (put in DP section under Constituency in Grammar)
		\item Mass (next to numeracy in morpho) % Move both of these to QP?
		\item Greetings (the appendix?)
		\item Enclitics (is this even relevant anymore? in the grammar section?)
	\end{enumerate}

\clearpage