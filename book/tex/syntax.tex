\chapter{Syntax}
	
	\section{Word order} 
		{\kurango} has a fixed SVO word order. The case assignment system is what makes the word order fixed (see Section \ref{case}).
		\subsection{Constituent order} % head final
		\subsection{Placement of adjuncts}
		% Consituent order (head final)
		% Also word order (SVO)
	\section{Morphological type} % Explain why you are putting this under syntax, include the figure from C+B, explain dialectical differences between this (conservative should be more isolating, more polysynthetic)

	\section{Case system} % accusative vs ergative
	\label{case}
		In \kurango , basic subject/object case assignment is based on word order. Arguments to left of the verb get ergative case, and arguments to the right of the verb get absolutive case. For {\D{erg}} and {\D{abs}}, there are no explicit case markers.



		\begin{example}
		\label{ex:case_wordorder_1}
			Ni'inakarana. [\textipa{niPinakaRana}]
			\gll ni- inakara -na
			2- bore -1
			\glt `You bore me.'
			\glend
		\end{example}

		\begin{example}
		\label{ex:case_wordorder_2}
			Na'inakarani. [\textipa{naPinakaRani}]
			\gll na- inakara -ni
			1- bore -2
			\glt `I bore you.'
			\glend
		\end{example}

		This ordering principle also allows us to categorize {\kurango} as a Fluid-S language, as arguments of intransitive verbs (S) pattern either with subjects of transitive verbs (S\textsubscript{A}) or objects of transitive verbs (S\textsubscript{O}) based on their position relative to the verb. Determining the way in which S patterns has to do with the semantics of the utterance: S\textsubscript{A} (marked with {\D{erg}}) has no entailment about the volition of the Agent in the action being performed, while S\textsubscript{O} (marked with {\D{abs}}) entails that the agent was volitional in the action being performed.

		\begin{example}
		\label{ex:case_erg}
			Nakari. [naka\R i]
			\gll na- kari
			1- sleep
			\glt `I sleep.' (by my own volition.)
			\glend
		\end{example}

		\begin{example}
		\label{ex:case_abs}
			Karina. [ka\R ina]
			\gll kari -na
			sleep -1
			\glt `I sleep.' (no entailment about my volition.)
			\glend
		\end{example}
	% Technically it's SVO, but, mention how it can be SV or VS in intransitive sentences (fluid S system)
		\subsection{Auxillary cases}
			In addition to the basic cases assigned by syntactic position, there are additional cases assigned by explicitly marked D- or P-heads. Below is a list and examples:
				\subsubsection{Genitive}
					Genitive (posessive) case is assigned by a postponed determiner \emph{-mi}.
					% example
				\subsubsection{Inessive}
					make a gloss for this
					% ``in the house''
				\subsubsection{Elative}
					% out of the house
				\subsubsection{Illative}
					% into the house
				\subsubsection{Adessive}
					% To the house
				\subsubsection{Ablative}
					% away from the house
				\subsubsection{Allative}
					% to the house

% Kurango:
	% Word order: SVO
	% No movement(adheres to VPiSH)
	% Head-final
	% Agglutinating with mild polysynthesis
	% Fluid S system

% Questions for Jenks:
	% Draw these trees:
		% Deep structure of English "He sees me"
		% Deep structure of English "The cat will give the mouse a scare"
		% Deep structure of an English yes/no question
		% Deep structure of an English wh-question
		% Deep structure of an English intransitive sentence
		% Deep structure of Negated english sentences
		% Deep structure of 'do' support
		% Deep structure of multi-clausal english sentences
		% The VPiSH sentence
	% Examples of languages without movement << Mandarin? Is there such a language?
	% Can a language be head-initial AND head-final based on different contextual rules?


% Head final under head initial (Think of german.)


% Subject: spec lil vp
% Object: spec VP

% Right branching = head on left

% Dative argument: Comp V


% DEEP STRUCTURE
	
