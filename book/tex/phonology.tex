\chapter{Phonology}
\label{chap:phono}

\section{Allophony}

Because \kurango's phonetic inventory is based around the idea of permutation, there is very little allophony. Most allophony in the language is the result of ameliorating coarticulatory strain.

	\subsection{Approximant formation}
		Intervocalic glottal stops undergo approximant formation into four possible allomorphs. The more ``progressive'' dialect changes from /\glot/ to [w] and [j], while the ``conservative'' dialect changes from /\glot/ to [\W] and [\M]. The conditioning environment for both set of changes is vowel laterality.

		\begin{figure}[H]
		\label{glides}
		\centering
		/\glot/ $\longrightarrow$ $\left\{ \begin{array}{l} \text{[j]}\\ \text{[\M]} \end{array}\right\}$ / V[-back]\uline V

		/\glot/ $\longrightarrow$ $\left\{ \begin{array}{l} \text{[w]}\\ \text{[\W]} \end{array}\right\}$ / V[+back]\uline V
		\caption{Phonological rules for formation of glides from glottal stops.}
		\end{figure}

 		In the sentence \emph{o'ati arimo} [\glot\OO wati\glot a\R im\OO]\footnote{This is not exactly a correct pronunciation. See the following section on affricates.}, the second (intervocalic) glottal stop in /\glot\OO\glot ati/ becomes [w](or [\W]). The glottal stop at the beginning of \emph{ari} also changes, but only because a second rule will apply to form a consonant cluster(see below). It is important to note that glottal stops at word boundaries that would not go on to be affected by the cluster formation rule do NOT become approximants (example: \emph{miro armio} /mi\R\OO\glot a\R im\OO/: [mi\R\OO\glot a\R im\OO]; *[mi\R wa\R im\OO]).

 	\subsection{Cluster formation}
 		When preceeded by stops and followed by sonorants, interconsonantal vowels are deleted. The deletion of this vowel forms a consonant cluster (marked with a tie bar) between the stop and the approximant.

 		This rule is in feeding order with the approximant formation rule, with the approximant formation rule applying first.

 		\begin{figure}[H]
		\label{cluster_form}
		\centering
		/V/ $\longrightarrow$ $\emptyset$ / C[-cont]\uline C[+cont, +son]

		\caption{Phonological rules for consonant cluster formation via vowel deletion.}
		\end{figure}

		This rule leaves our sentence \emph{o'ati arimo} from above with the final phonetic form as [owatjarimo]

		I have chose to call this rule ``Cluster formation'' instead of ``Deletion of post-stop, pre-approximant vowels'' because the primary purpose of the rule is to form consonant cluster that are easier to phonate than sequences with glottal stops in them.

\section{Suprasegmental features (Prosody)}
	\subsection{Rhythm}
	\subsection{Tone}
	\subsection{Stress}
	\subsection{Intonation}


  	% Also glide formation for w
  	% i/a > j (Non-back vowels --- functionally front vowels)
  	% o/u > w (Back vowels)