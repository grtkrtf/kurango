\chapter{Questions}

\section{Syntax-related questions for Professor Jenks}
	\subsection{Trees}
	Let's draw trees for the following examples.
		\begin{enumerate}
			\item A simple intransitive sentence:
				\begin{example}
				\label{simple_intrans}
					Nakari. [na.\stress ka.\R i]
					\gll na- kari
					1- sleep
					\glt `I sleep.'
					\glend
				\end{example}
			\item A simple intransitive sentence:
				\begin{example}
				\label{simple_intrans_erg}
					Karina. [ka.\stress\R i.na]
					\gll kari -na
					sleep -1
					\glt `I sleep. (No entailment about my volition)'
					\glend
				\end{example}

			\item A simple passivized intransitive:
				\begin{example}
				\label{intrans_pass}
					Karirhu. [ka.\stress\R i.\G u]
					\gll ka\R i -\G u
					sleep -\D{pass}
					\glt `Sleeping happened.'
					\glend
				\end{example}

			\item A simple transitive sentence:
				\begin{example}
				\label{simple_trans}
					Nivoruna. [ni.\stress\B\OO.\R u.na]
					\gll ni- \B\OO\R u -na
					2- please -1
					\glt `You please me.'
					\glend
				\end{example}

			\item A simple passivized transitive sentence:
				\begin{example}
				\label{trans_pass}
					Vorurhuna. [\B\OO.\stress\R u.\G u.na]
					\gll \B\OO ru -\G u -na
					please -\D{pass} -1
					\glt `I am pleased.'
					\glend
				\end{example}

			\item A ``simple'' ditransitive sentence:
				\begin{example}
				\label{simple_ditrans}
					Nufu'arina faavukoo. [nu.\stress\F u.wa.\R i.na \F a\len.\stress\B u.k\OO\len]
					\gll nu- \F u- \glot ari -na fa\len\B u -k\OO\len
					3- \D{inan}- \D{v:motion} -1 home -\D{dir:prox}
					\glt `It follows me home.'
					\glend
				\end{example}

			\item Structure of adjectives:
				\begin{example}
				\label{adj}
					Gaka si. [\stress ga.ka si]
					\gll ga -ka si
					emotional -\D{emo:content} \D{cop}
					\glt `I am content.'
					\glend
				\end{example}

			\item Structure of PPs:
				\begin{example}
				\label{pp}
					Riitii o'atiwakoo ro! [{\stress\R i\len.ti\len} \glot\OO.\stress wa.ti.\W a.{k\OO\len} \R\OO]
					\gll {\R i\len ti\len} \glot\OO\glot ati -\W a -{k\OO\len} \R\OO
					look dog -that -\D{dir:prox} \interr
					\glt `Look at that dog!'
					\glend
				\end{example}

			\item Structure of adverbs:
				\begin{example}
				\label{adv}
					Fivu simo adarazu. [\stress\F i.\B u \stress si.{m\OO} \glot a.\stress da.\R a.zu]
					\gll \F i\B u si {-m\OO} \glot ada\R a -zu
					house \D{cop} -\D{pst} order -\D{deriv:v.to.adv}
					\glt `Her house was orderly.'
					\glend
				\end{example}

			\item Subordinate clauses:
				\begin{example} % Probably wrong.
				\label{sub_emo} 
					Gatitixu si gangawufupataduzu si. [ga.\stress ti.ti.xu si ga.\stress\N a.\W u.\F u.pa.ta.du.zu si]
					\gll ga -titi -xu si ga -\N a -\W u -\F u -pa -taduzu si
					\D{emo} -\D{emo:strong.affection} -\D{targ:anim} \D{cop} \D{emo} -\D{neg} -\D{emo:confidence} -\D{targ:null} -\D{dur:short} -\D{sub} \D{cop}
					\glt `I love someone, but I am also scared.'
					\glend
				\end{example}
				
				\begin{example} % Tense marking in this example?
				\label{sub}
					Nakuramo miro oramotaduzu [{na.\stress ku.\R a.m\OO} {\stress mi.\R\OO} \glot\OO.\stress\R a.m\OO .ta.du.zu]
					\gll na- ku\R a {-m\OO} {mi\R\OO} \glot\OO\R a -{m\OO} -taduzu
					1- think -\D{pst} cat punch -\D{pst} -\D{sub}
					\glt `I thought about the cat that I punched.'
					\glend
				\end{example}

			\item Interaction of modals and tense:
				\begin{example} % May also be nakarimokanga. ?? Order of PST and PERF?
				\label{modals_and_tense}
					Nakarikangamo. [na.\stress ka.\R i.ka.\N a.m\OO] 
					\gll na- kari -ka\N a -m\OO
					1- sleep -\D{perf} -\D{pst}
					\glt `I had slept.'
					\glend
				\end{example}

			\item Polar question:
				\begin{example}
				\label{polar}
					Na'arigaa faavugoo ri? [na.\stress ja.\R i.{ga\len} \F a\len .\stress\B u.{g\OO\len} \R i]
					\gll na- \glot a\R i -{ga\len} \F a\len\B u -{g\OO\len} \R i
					1- go -\D{fut} house -\D{dir:dist} \interr
					\glt `Are you going home?'
					\glend
				\end{example}

			\item wh- questions:
				\begin{example}
				\label{wh1}
					Mirowa numunguumo o'ati ri? [mi.\stress\R\OO.\W a {nu.\stress mu.\N u\len.m\OO} \glot\OO.\stress wa.ti \R i]
					\gll {mi\R\OO} -\W a {numu\N u\len} -{m\OO} \glot\OO\glot ati \R i
					cat -that smell -\D{pst} dog \D{interr}
					\glt `Which cat smelled the dog?'
					\glend
				\end{example}

				\begin{example}
				\label{wh2}
					Miro numunguumo o'atiwa ri? [\stress mi.\R\OO {nu.\stress mu.\N u\len.m\OO} \glot\OO.\stress wa.ti.\W a \R i]
					\gll {mi\R\OO} {numu\N u\len} -{m\OO} \glot\OO\glot ati -\W a \R i
					cat smell -\D{pst} dog -that \D{interr}
					\glt `The cat smelled which dog?'
					\glend
				\end{example}

			\item Negation:
				\begin{example}
				\label{negation}
					Nakaringa. [na.\stress ka.\R i.\N a]
					\gll na- kari -\N a
					1- sleep -\negative
					\glt `I do not sleep.'
					\glend
				\end{example}

			\item Causatives:
				\begin{example} % change the order maybe?
				\label{caus}
					Namotivagaanu. [na.\stress m\OO.ti.\B a.ga\len.nu]
					\gll na- m\OO ti -\B a -gaa -nu
					1- die -{\caus} -{\fut} -3
					\glt `I will kill it.'
					\glend
				\end{example}

			\item More modals:
				\begin{example}
				\label{modals_2}  % gloss be.able.to better?
					Nasaru karipa. [na.\stress sa.\R u ka.\stress\R i.pa]
					\gll na- saru karipa
					1- be.able.to nap 
					\glt `I can nap.'
					\glend
				\end{example}

			\item A reflexive:
				\begin{example}
				\label{reflexive}
					Natigo'imuzu. [na.\stress ti.gji.mu.zu]
					\gll na- ti- go\glot i -muzu
					1- \D{pl}- desire -\D{refl}
					\glt `We want each other.'
					\glend
				\end{example}

			\item Another wh-question:
				\begin{example}
				\label{wh_2}
					Anganimi siri? [\glot a.\stress\N a.ni.mi \stress si.\R i]
					\gll \glot a\N a -ni -mi si -\R i
					name -2 -{\poss} {\cop} -\interr
					\glt `What is your name?'
					\glend
				\end{example}
		\end{enumerate}

	\subsection{Other questions}
		\begin{enumerate}
			\item a `minimal pair' of a lang with do support and a lang without do support
			\item do all languages syntactically differentiate polar/wh?
		\end{enumerate}



%------------------------------------------------------------------%
% Questions for Jenks:
	% Draw these trees:
		% Deep structure of English "He sees me" | SIMLPE TRANSITIVE
		% Deep structure of English "The cat will give the mouse a scare" | SIMPLE DITRANSITIVE
		% Deep structure of an English yes/no question | YN
		% Deep structure of an English wh-question | WH
		% Deep structure of an English intransitive sentence | SIMPLE INTRANSITIVE
		% Deep structure of Negated english sentences
		% Deep structure of 'do' support
		% Deep structure of multi-clausal english sentences
		% The VPiSH sentence
	% Examples of languages without movement << Mandarin? Is there such a language?
	% Can a language be head-initial AND head-final based on different contextual rules?


% Head final under head initial (Think of german.)


% Subject: spec lil vp
% Object: spec VP

% Right branching = head on left

% Dative argument: Comp V


% DEEP STRUCTURE



%%%%%%%%
% jenks session 12/6/15
% Typologically it's weird to have postpositions in an svo langauge.
% ro is an aspect phrase that causes head initial to become head final
	% ro is in C
	% aspect should occur between head and verb
	% aspect should be close to verb than tense
% need a wh suffix
	% Think more about wh questions.
% distance from verb: Tense > neg > aspect


% Y DP: it seems like it's organized with a lexical head that has functional structure above it. We know clauses are organzed this way bc head movement.
% D head is inflecting for case in German, not the noun. case is a property of where the dp shows up in the clause!!!!!
% In italian: head movement from N to D
% gabardi 1993

